\section{Proofs and justification}
\label{sec:justification}

Some implementation details require further justification and explanations.

\subsection{Ray operations (rop)}
\subsubsection{Reflection formula derivation}
We derive the reflection formula. See the illustration \cref{fig:reflect}.
Let: \begin{itemize}
\item The surface normal $\overrightarrow{N}$ at the ray intersection.
\item The incident ray unit vector $\overrightarrow{i}$.
\item The reflected ray unit vector $\overrightarrow{r}$.
\item $\theta_i$ the (unsigned) angle between $\overrightarrow{N}$ and
      $\overrightarrow{i}$.
\item $\theta_r$ the (unsigned) angle between $\overrightarrow{N}$ and
      $\overrightarrow{r}$.
\end{itemize}

We want to compute $\overrightarrow{r}$ as a function of $\overrightarrow{i}$
and $\overrightarrow{N}$.  We assume the laws of reflection, which essentially
state that light behaves as a billiard ball in its geometrical interaction with
the surface.  We present several ways of looking at the problem. In our
opinion, these are not so much derivations as different restatements of the
laws of reflection.

\paragraph{Bounce derivation}
The effect of the light bouncing off the surface can be stated in the following
fashion. The component of $\overrightarrow{i}$ colinear to $\overrightarrow{N}$
is inverted by the reflection. The other components of $\overrightarrow{i}$ stay
the same. This immediately gives the tractable relation.

\begin{equation}
\overrightarrow{r} = \overrightarrow{i} - 2 \cdot (\overrightarrow{N} \cdot
\overrightarrow{i}) \cdot \overrightarrow{N}
\end{equation}

\paragraph{Geometric construction}
The ray reflection may be viewed as a geometric construction based on the
laws of reflection (see \cite{Glassner:1989} p.291 \cite{Comninos:2010}
p.335). We draw the geometric view of the problem on
\cref{fig:reflect-geometry}.

\begin{figure} \caption{\label{fig:reflect-geometry} Illustration for the
geometric construction of the reflect operation.}
\includesvg[height=.2\textheight, width=.9\textwidth, keepaspectratio]
           {images/shape/reflect-geometry.svg}
\end{figure}

\paragraph{Algebraic derivation}
We produce a derivation similar to the one included in \cite{Glassner:1989}
(p.131).

\begin{itemize}
\item The reflected ray is in the plane spanned by $\overrightarrow{i}$ and
      $\overrightarrow{N}$ (\emph{plane of incidence}).
\item $\theta_i = \theta_r$
\item Except in the degenerate case $\overrightarrow{i} = - \overrightarrow{N}$,
$\overrightarrow{r} \neq - \overrightarrow{i}$. Ie the reflected ray does not
retrace on the incident ray.
\end{itemize}

Translated algebraically, these statements suffice to determine a formula for
$\overrightarrow{r}$. Let $(\alpha, \beta)$ both in $\mathbb{R}$. The reflected
ray in the incidence plane must be expressed as:

\begin{equation}
\overrightarrow{r} = \alpha \cdot \overrightarrow{i} +
                     \beta \cdot \overrightarrow{N}
\end{equation}

We now add the remaining constraints on $\overrightarrow{r}$ in order to narrow
down expressions for $\alpha$ and $\beta$.

\begin{equation} \begin{split}
& \theta_i = \theta_r \\
\implies & \cos(\theta_i) = \cos(\theta_r) \\
\implies & - \overrightarrow{i} \cdot \overrightarrow{N} =
       \overrightarrow{N} \cdot \overrightarrow{r}
\end{split} \end{equation}

\begin{equation} \begin{split}
- \overrightarrow{i} \cdot \overrightarrow{N} &=
\overrightarrow{N} \cdot
(\alpha \cdot \overrightarrow{i} + \beta \cdot \overrightarrow{N}) \\
&= \alpha \cdot \overrightarrow{i} \cdot \overrightarrow{N} +
   \beta \cdot \overrightarrow{N} \cdot \overrightarrow{N} \\
&= \alpha \cdot \overrightarrow{N} \cdot \overrightarrow{i} + \beta
\end{split} \end{equation}

Hence the following expression for $\beta$.

\begin{equation} \label{eq:reflect-beta-expr}
\beta = - (\alpha + 1) \cdot \overrightarrow{N} \cdot \overrightarrow{i}
\end{equation}

Another constraint we exploit is that $\overrightarrow{r}$ is a unit vector.

\begin{equation} \begin{split}
& \abs{\overrightarrow{r}} = 1 \\
\implies& \abs{\alpha \cdot \overrightarrow{i} +
                \beta \cdot \overrightarrow{N}} = 1 \\
\implies& (\alpha \cdot i_x + \beta \cdot N_x)^2 +
          (\alpha \cdot i_y + \beta \cdot N_y)^2 +
          (\alpha \cdot i_z + \beta \cdot N_z)^2 = 1 \\
\implies& \alpha^2 ({i_x}^2 + {i_y}^2 + {i_z}^2) +
          \beta^2 ({N_x}^2 + {N_y}^2 + {N_z}^2) + \\
          &2 \cdot \alpha \cdot \beta 
          (i_x \cdot N_x + i_y \cdot N_y + i_z \cdot N_z) = 1 \\
\implies& \alpha^2 + \beta^2 + 2 \cdot \alpha \cdot \beta \cdot
          \overrightarrow{N} \cdot \overrightarrow{i} = 1
\end{split} \end{equation}

Now we plug the expression for $\beta$ \cref{eq:reflect-beta-expr}.

\begin{equation} \begin{split}
\implies& \alpha^2 + (\alpha + 1)^2
(\overrightarrow{N} \cdot \overrightarrow{i})^2
- 2 \cdot \alpha (\alpha + 1) \cdot
(\overrightarrow{N} \cdot \overrightarrow{i})^2 = 1 \\
\implies& \alpha^2 (1 - 2 (\overrightarrow{N} \cdot \overrightarrow{i})^2
+ (\overrightarrow{N} \cdot \overrightarrow{i})^2) + \\
& \alpha (-2\cdot (\overrightarrow{N} \cdot \overrightarrow{i})^2
        +2 \cdot (\overrightarrow{N} \cdot \overrightarrow{i})^2) +
(\overrightarrow{N} \cdot \overrightarrow{i})^2 = 1 \\
\implies& \alpha^2 \cdot (1 - (\overrightarrow{N} \cdot \overrightarrow{i})^2)
+ (\overrightarrow{N} \cdot \overrightarrow{i})^2 = 1 \\
\implies& \alpha^2 = \frac{1 - (\overrightarrow{N} \cdot \overrightarrow{i})^2}
                          {1 - (\overrightarrow{N} \cdot \overrightarrow{i})^2}
\end{split} \end{equation}

The degenerate case $\overrightarrow{i} = - \overrightarrow{N}$, which
leads to $1 - (\overrightarrow{N} \cdot \overrightarrow{i})^2 = 0$, is excluded.
The answer in the degenerate case is $\overrightarrow{r} = -
\overrightarrow{i}$. We continue the main derivation path with:

\begin{equation}
\alpha = \pm 1
\end{equation}

The case $\alpha = -1$ leads to $\beta = 0$ and $\overrightarrow{r} = -
\overrightarrow{i}$. We excluded this case by hypothesis.  The remaining answer
is $\alpha = + 1$.

Summarizing:

\begin{equation} \begin{cases}
\alpha = 1 \\
\beta = -2 \cdot \overrightarrow{N} \cdot \overrightarrow{i}
\end{cases} \end{equation}

This leads us to the expression for $\overrightarrow{r}$.

\begin{equation}
\overrightarrow{r} = \overrightarrow{i} - 2 \cdot (\overrightarrow{N} \cdot
\overrightarrow{i}) \cdot \overrightarrow{N}
\end{equation}

$\square$

\subsubsection{Refraction}
We derive the refraction formula vectorial form from the laws of refraction.
Then we check the equivalence of Bec's formula with the derived refraction
formula.

\paragraph{Algebraic derivation}
We work through an algebraic derivation of the vectorial expression for
the refraction operation. Similar derivations may be found in
\cite{Glassner:1989} (p.288), originally by Whitted \cite{Whitted:2005}
and Heckbert \cite{Heckbert:1984}.

The quantities involved in the problem are (illustrated on
\cref{fig:refract-derivation}):

\begin{itemize}
\item $n_1$ the incident medium refraction index,
\item $n_2$ the output medium refraction index,
\item $\overrightarrow{i}$ the unit incident ray direction,
\item $\overrightarrow{N}$ the unit surface normal vector,
\item $\overrightarrow{t}$ the unit refracted ray direction,
\item $\theta_i$ the acute angle between $-\overrightarrow{i}$
and $\overrightarrow{N}$,
\item $\theta_t$ the angle between $-\overrightarrow{N}$ and
$\overrightarrow{t}$.
\end{itemize}

\cref{fig:refract-derivation}
\begin{figure} \caption{\label{fig:refract-derivation} Quantities involved
in the ray refraction operation.}
\includesvg[height=.2\textheight, width=.9\textwidth, keepaspectratio]
           {images/shape/refract-derivation.svg}
\end{figure}

We take for granted a number of assumptions related to the law of refraction
\cite{wiki:snell-refraction}:

\begin{itemize}
\item $\overrightarrow{t}$ is in the plane spanned by $\overrightarrow{i}$
and $\overrightarrow{N}$ (\emph{plane of incidence}),
\item Defining quadrants in the plane of incidence with respect to
$\overrightarrow{N}$ and the origin, $\overrightarrow{t}$ is in
the same quadrant as $\overrightarrow{i}$,
\item $n_1 \cdot \sin(\theta_i) = n_2 \cdot \sin(\theta_t)$
\item In the particular case where $-\overrightarrow{i} = \overrightarrow{N}$,
      the refracted ray is $\overrightarrow{t} = - \overrightarrow{N}$.
\end{itemize}

We remind some relations and define some shorthands used for conciseness.

\begin{equation} \begin{cases}
- \overrightarrow{N} \cdot \overrightarrow{i} = \cos(\theta_i) = c_i \\
- \overrightarrow{N} \cdot \overrightarrow{t} = \cos(\theta_t) = c_t \\
\sin(\theta_i) = s_i \\
\sin(\theta_t) = s_t \\
\frac{n_1}{n_2} = n_r
\end{cases} \end{equation}

Since $\overrightarrow{t}$ is in the incidence plane, it may be written as:

\begin{equation}
\overrightarrow{t} = \alpha \cdot \overrightarrow{i}
                     + \beta \cdot \overrightarrow{N}
\end{equation}

The other hypotheses provide the following constraints, which we will use
in order to solve for $\alpha$ and $\beta$.

\begin{equation} \begin{cases}
n_1 \cdot s_i = n_2 \cdot s_t \\
\abs{\overrightarrow{t}} = 1
\end{cases} \end{equation}

We may express a relation with the help of $c_t$:

\begin{equation} \begin{split}
c_t &= - \overrightarrow{N} \cdot \overrightarrow{t} \\
&= - \overrightarrow{N} \cdot
   (\alpha \cdot \overrightarrow{i} + \beta \cdot \overrightarrow{N}) \\
&= - \alpha \cdot \overrightarrow{N} \cdot \overrightarrow{i} - \beta \\
&= \alpha \cdot c_i - \beta
\end{split} \end{equation}

Thus,

\begin{equation}
\beta = \alpha \cdot c_i - c_t
\end{equation}

The $\overrightarrow{t}$ normalization constraint gives us another expression:

\begin{equation} \begin{split}
\abs{\overrightarrow{t}} &= \abs{\alpha \cdot \overrightarrow{i}
                                 + \beta \cdot \overrightarrow{N}} \\
&= (\alpha \cdot i_x + \beta \cdot N_x)^2 +
   (\alpha \cdot i_y + \beta \cdot N_y)^2 +
   (\alpha \cdot i_z + \beta \cdot N_z)^2 \\
&= \alpha^2 \cdot ({i_x}^2 + {i_y}^2 + {i_z}^2) +
   \beta^2 \cdot ({N_x}^2 + {N_y}^2 + {N_z}^2) + \\
   & 2 \cdot \alpha \cdot \beta \cdot
     (i_x \cdot N_x + i_y \cdot N_y + i_z \cdot N_z) \\
&= \alpha^2 + \beta^2 + 2 \cdot \alpha \cdot \beta \cdot \overrightarrow{i}
   \cdot \overrightarrow{N} \\
&= \alpha^2 + \beta^2 - 2 \cdot \alpha \cdot \beta \cdot c_i
\end{split} \end{equation}

Now we plug in the expression for $\beta$.

\begin{equation} \begin{split}
\abs{\overrightarrow{t}} &=
\alpha^2 + (\alpha \cdot c_i - c_t)^2 - 2 \cdot \alpha \cdot c_i \cdot
(\alpha \cdot c_i - c_t) \\
&= \alpha^2 + \alpha^2 \cdot {c_i}^2 + {c_t}^2
- 2 \cdot \alpha \cdot c_i \cdot c_t - 2 \cdot \alpha^2 \cdot {c_i}^2
+ 2 \cdot \alpha \cdot c_i \cdot c_t \\
&= \alpha^2 - \alpha^2 \cdot {c_i}^2 + {c_t}^2 \\
&= 1
\end{split} \end{equation}

Thus, making use of the Snell formula, and excluding the $c_i = 1$ case
for which the refracted ray is just $- \overrightarrow{N}$,

\begin{equation} \begin{split}
& \alpha^2 = \frac{1 - {c_t}^2}{1 - {c_i}^2} = \frac{{s_t}^2}{{s_i}^2} =
  {n_r}^2 \\
& \iff \alpha = \pm n_r
\end{split} \end{equation}

By the $\overrightarrow{t}$ quadrant hypothesis, the component along
$\overrightarrow{i}$ must be positive, hence $\alpha > 0 \implies \alpha = +
n_r$.

We have now solved for the scalar magnitudes of $\overrightarrow{t}$.

\begin{equation} \begin{cases}
\alpha = n_r \\
\beta = n_r \cdot c_i - c_t
\end{cases} \end{equation}

And $\overrightarrow{t}$ may be expressed as:

\begin{equation}
\overrightarrow{t} = n_r \cdot \overrightarrow{i}
 + (n_r \cdot c_i - c_t) \cdot \overrightarrow{N}
\end{equation}

All that is left is to express $c_t$ with respect to input problem quantities.
The case of \gls{TIR} appears when ${n_r}^2 \cdot {s_i}^2 > 1$ and is excluded
from the remainder of the derivation.

\begin{equation} \begin{split}
c_t &= \sqrt{1 - {s_t}^2} \\
&= \sqrt{1 - {n_r}^2 \cdot {s_i}^2} \\
&= \sqrt{1 - {n_r}^2 \cdot (1 - {c_i}^2)}
\end{split} \end{equation}

Thus the expression for $\overrightarrow{t}$ with respect to the problem's input
quantities is:

\begin{equation} \label{eq:refraction-derived}
\overrightarrow{t} = n_r \cdot \overrightarrow{i} -
\left(n_r \cdot \overrightarrow{N} \cdot \overrightarrow{i}
+ \sqrt{1 - {n_r}^2 \cdot \left(1 - 
             \left(\overrightarrow{N} \cdot \overrightarrow{i}\right)^2\right)}
\right)
\cdot \overrightarrow{N}
\end{equation}

$\square$

\paragraph{Bec's formula validation}
We perform a check of Bec's formula's \cref{eq:bec-formula} validity with
respect to the derived refraction formula \cref{eq:refraction-derived}. Bec's
formula is expressed as:

\begin{equation}
\overrightarrow{t} = n_r \cdot \overrightarrow{i} +
(w - \sqrt{1 + c_{2m}}) \cdot \overrightarrow{N} 
\end{equation}

With:

\begin{equation} \begin{cases}
w &= - n_r \cdot \overrightarrow{i} \cdot \overrightarrow{N} \\
c_{2m} &= (w - n_r) \cdot (w + n_r)
\end{cases} \end{equation}

While the formula from our derivation is expressed as:

\begin{equation}
\overrightarrow{t} = n_r \cdot \overrightarrow{i} -
\left(n_r \cdot \overrightarrow{N} \cdot \overrightarrow{i}
+ \sqrt{1 - {n_r}^2 \cdot \left(1 - 
             \left(\overrightarrow{N} \cdot \overrightarrow{i}\right)^2\right)}
\right)
\cdot \overrightarrow{N}
\end{equation}

We may re-express it as:

\begin{equation}
\overrightarrow{t} = n_r \cdot \overrightarrow{i} +
\left(- n_r \cdot \overrightarrow{N} \cdot \overrightarrow{i}
- \sqrt{1 + \gamma} \right) \cdot \overrightarrow{N}
\end{equation}

We may check Bec's formula by proving that $\gamma = c_{2m}$.

\begin{equation} \begin{split}
c_{2m} &= (w - n_r) \cdot (w + n_r) = w^2 - {n_r}^2 \\
       &= {n_r}^2 \cdot
          \left( \overrightarrow{i} \cdot \overrightarrow{N} \right)^2
          - {n_r}^2 \\
       &= -{n_r}^2 \cdot
           \left(1 - \left(\overrightarrow{i} \cdot
                     \overrightarrow{N}\right)^2\right) \\
       &= \gamma
\end{split} \end{equation}

Thus we find Bec's formula to be mathematically equivalent to the refraction
formula we derived.

\subsection{transfer operation}
We work through the formula for applying the transfer operation to a ray.
We reuse the notations in the definition (\cref{sec:transfer}). The transfer
operation is the composition of a change of basis and an intersection with
the new local plane.

\paragraph{Change of basis}
The initial ray $(P_1, \overrightarrow{V_1})$ is expressed in LCS1 coordinates.
We express it in LCS2 with $(P_2, \overrightarrow{V_2})$. The LCS2 basis vectors
are obtained by a rotation of the LCS1 basis vectors. We express everything
in LCS1 coordinates up to the computation of $(P_2, \overrightarrow{V_2})$.

\begin{equation} \begin{cases}
\hat{x_2} = B \cdot \hat{x_1} \\
\hat{y_2} = B \cdot \hat{y_1} \\
\hat{z_2} = B \cdot \hat{z_1}
\end{cases} \end{equation}

The origin $A_2$ of LCS2 is obtained by a translation of $A_1$ along
the new, rotated, coordinates.

\begin{equation}
D = \begin{bmatrix} D_x \\ D_y \\ D_z \end{bmatrix}
\end{equation}

\begin{equation} \begin{split}
A_2 &= A_1 + D_x \cdot \hat{x_2} + D_y \cdot \hat{y_2} + D_z \cdot \hat{z_2} \\
    &= D_x \cdot \hat{x_2} + D_y \cdot \hat{y_2} + D_z \cdot \hat{z_2} \\
\end{split} \end{equation}

Let us remember the columns of $B$ are the basis vectors of LCS2 expressed
in LCS1.

\begin{equation}
B = \left[ \hat{x_2}, \hat{y_2}, \hat{z_2} \right]
\end{equation}

Thus, we can simplify the expression of $A_2$.

\begin{equation}
A_2 = B \cdot D
\end{equation}

The coordinates of $P_2$ are expressed thanks to a dot product with the basis
vectors of LCS2.

\begin{equation} \begin{cases}
x_2 = (P - A_2) \cdot \hat{x_2} \\
y_2 = (P - A_2) \cdot \hat{y_2} \\
z_2 = (P - A_2) \cdot \hat{z_2}
\end{cases} \end{equation}

Again, this can be viewed as the following matrix-vector multiplication.

\begin{equation} \begin{split}
P_2 &= B^\top \cdot (P - A_2) \\
    &= B^\top \cdot (P - B \cdot D) \\
    &= B^\top \cdot P - D
\end{split} \end{equation}

Similarly, $\overrightarrow{V_2}$ may be expressed in LCS2.

\begin{equation} \begin{cases}
l_2 = \overrightarrow{V_1} \cdot \hat{x_2} \\
m_2 = \overrightarrow{V_1} \cdot \hat{y_2} \\
n_2 = \overrightarrow{V_1} \cdot \hat{z_2}
\end{cases} \end{equation}

Which can be simplified as the following relation.

\begin{equation}
\overrightarrow{V_2} = B^\top \cdot \overrightarrow{V_1}
\end{equation}

\paragraph{Intersection with the new local plane}
The ray characterized by $(P_2, \overrightarrow{V_2}$ is intersected
with the LCS2 $z = 0$ plane. Every operation happens in LCS2 coordinates.
The intersection condition between the ray parametrized by $t$ and the
plane is expressed as follows.

\begin{equation}
z_2 + t \cdot n_2 = 0
\end{equation}

We treat the case $n_2 = 0$ as an error case. It corresponds to the ray
being parallel to the plane.

\begin{equation}
t = - \frac{z_2}{n_2}
\end{equation}

We simply propagate point $P_2$ up to the intersection $P_3$.

\begin{equation}
P_3 = \begin{bmatrix}
x_2 + t \cdot l_2 \\
y_2 + t \cdot m_2 \\
0
\end{bmatrix}
\end{equation}

The ray vector $\overrightarrow{V_2}$ does not undergo any operation, hence
$\overrightarrow{V_3} = \overrightarrow{V_2}$.

\subsection{General expression for a shape normal vector}
Given either an explicit formula ($z = f(x, y)$), or an implicit
formula ($F(x, y, z) = 0$) representing a surface, we show how
to compute the unit normal vector at a given point. Additionally
we orient the normal vector in the opposite half-plane to an incident
ray with a vector component $n$.

\paragraph{From an explicit formula}
Given a shape in its \gls{LCS} defined by an equation $z = f(x, y)$,
we can compute the unit normal vector at point $(x, y)$ using
\cref{eq:gen-normal} \cite{mathworld:normal-vector}. 

\begin{equation} \label{eq:gen-normal}
\overrightarrow{N} =
\frac{\textrm{sign}(n)}
     {\sqrt{1 + \left(\frac{\partial f}{\partial x}(x, y)\right)^2 +
                \left(\frac{\partial f}{\partial y}(x, y)\right)^2}}
\cdot
\begin{bmatrix}
\frac{\partial f}{\partial x}(x, y) \\
\frac{\partial f}{\partial y}(x, y) \\
-1
\end{bmatrix}
\end{equation}

\paragraph{From an implicit formula}
A normal vector may also be defined using an implicit surface definition of
the form $F(x, y, z) = 0$ (\cite{wiki:normal}, \cite{Welford:1986} section 4.6).
A normal vector at a point on the surface is simply given by the gradient
$\nabla F$ evaluated at this point.

\begin{equation}
\nabla F =
\begin{bmatrix}
\frac{\partial F}{\partial x} \\
\frac{\partial F}{\partial y} \\
\frac{\partial F}{\partial z}
\end{bmatrix}
\end{equation}

Thus the unit normal vector oriented opposite to $n$ is given by:

\begin{equation}
\overrightarrow{N} = - \frac{\nabla F (x, y, z)}
                          {\abs{\nabla F(x, y, z)}} \cdot
  \textrm{sign}(n) \cdot
  \textrm{sign}\left(\frac{\partial F}{\partial z}(x, y, z)\right)
\end{equation}

\subsection{shape: standard}
The \lstinline{standard} shape is defined with the formula in
\cref{eq:standard-z}. This formula may be found in \cite{Welford:1986}
(section 4.7) and in \cite{Greynolds:2002} \footnote{Note the use of this kind
of surface in optics dates back to at least Kepler in this present form and all
the way back to the Greeks for the general idea.}.  We explore its link with
quadrics and try to provide a rationale for how it was defined from quadrics.

Note we freely convert between $R$ and $c$, and $r$ and $(x, y)$ in
the derivations.

\subsubsection{Range of definition}
The range in $r$ for which the \lstinline{standard} shape is defined
depends on $k$. The shape is defined when,

\begin{equation} \begin{split}
& 1 - (k+1) \cdot c^2 \cdot r^2 \geq 0 \\
\iff & (k+1) \cdot c^2 \cdot r^2 \leq 1
\end{split} \end{equation}

For $k \leq -1$, the validity condition is met for $r \in \mathbb{R}$.
For $k > -1$, the condition is met when,

\begin{equation} \begin{split}
& r^2 \leq \frac{1}{c^2 \cdot (k+1)} \\
\iff & r \leq \frac{\abs{R}}{\sqrt{k+1}}
\end{split} \end{equation}

\subsubsection{Link with quadrics}
We exhibit the link between the \lstinline{standard} surface and the
more general quadrics expressions.

\paragraph{General quadrics form}
An expression for general quadrics can be written as \cite{wiki:quadric}.

\begin{equation}
A x^2 + B y^2 + C z^2 + D x y + E y z + F x z + G x + H y + I z + J = 0
\end{equation}

With,
\begin{itemize}
\item $A, B, C, D, E, F, G, H, I, J$ all in $\mathbb{R}$,
\item At least one of $A$, $B$ or $C$ is non-zero.
\end{itemize}

We can easily see to which family of quadrics the points defined by the
\lstinline{standard} altitude formula \emph{belong to}. We start with
the altitude expression and re-express it in the general quadrics form.

\begin{equation} \begin{split}
&z = \frac{c \cdot r^2}{1 + \sqrt{1 - (k + 1) \cdot c^2 \cdot r^2}} \\
\implies & z + z \cdot \sqrt{1 - (k+1) \cdot c^2 \cdot r^2} = c \cdot r^2 \\
\implies & z - c \cdot r^2 = - z \cdot \sqrt{1 - (k + 1) \cdot c^2 \cdot r^2} \\
\implies & \frac{c \cdot r^2}{z} - 1 = \sqrt{1 - (k + 1) \cdot c^2 \cdot r^2} \\
\implies & \frac{c^2 \cdot r^4}{z^2} + 1 - \frac{2 \cdot c \cdot r^2}{z} =
           1 - (k + 1) \cdot c^2 \cdot r^2 \\
\implies & \frac{c \cdot r^2}{z^2} - \frac{2}{z} = - (k + 1) \cdot c \\
\implies & (k + 1) \cdot z^2 - \frac{2}{c} \cdot z + r^2 = 0 \\
\implies & x^2 + y^2 + (k + 1) \cdot z^2 - 2 R \cdot z = 0
\end{split} \end{equation}

We readily see this quadric can be defined from the general quadric expression
by setting:

\begin{equation} \begin{cases}
A = B = 1 \\
C = k + 1 \\
I = -2 R \\
D = E = F = G = H = J = 0
\end{cases} \end{equation}

\paragraph{Normal form}
We operate the suitable change of coordinates in order to reduce the quadric
expression to a so-called \emph{normal form} (\cite{wiki:quadric},
\cite{Venit:2008} p.384) and better identify the types of quadrics we are
working with.

First, we treat the case $k=-1$.
The quadric takes the following form.

\begin{equation} \begin{split}
&x^2 + y^2 - 2 R \cdot z = 0 \\
\iff &z = \frac{c}{2} \cdot x^2 + \frac{c}{2} \cdot y^2
\end{split} \end{equation}

This form is a \emph{circular paraboloid}.

Then, we treat the case $k \neq -1$. Let $\varepsilon = k + 1$.

\begin{equation} \begin{split}
& x^2 + y^2 + \varepsilon \cdot z^2 - 2 R \cdot z = 0 \\
\iff & x^2 + y^2 + \varepsilon \cdot \left( z^2 - \frac{2R}{\varepsilon} \cdot z\right) = 0 \\
\iff & x^2 + y^2 + \varepsilon \cdot \left( z - \frac{R}{\varepsilon} \right)^2 - \frac{R^2}{\varepsilon}
       = 0
\end{split} \end{equation}

We operate the change of coordinate $z' = z - \frac{R}{\varepsilon}$.

\begin{equation} \begin{split}
& x^2 + y^2 + \varepsilon \cdot z'^2 - \frac{R^2}{\varepsilon} = 0 \\
\iff & x^2 \cdot \frac{\varepsilon}{R^2} + y^2 \cdot \frac{\varepsilon}{R^2} +
       z'^2 \cdot \frac{\varepsilon^2}{R^2} - 1 = 0 \\
\iff & x^2 \cdot \frac{\textrm{sign}(\varepsilon) \cdot \abs{\varepsilon}}{R^2} +
       y^2 \cdot \frac{\textrm{sign}(\varepsilon) \cdot \abs{\varepsilon}}{R^2} +
       z'^2 \cdot \frac{\varepsilon^2}{R^2} - 1 = 0 \\
\iff & x^2 \cdot \frac{\abs{\varepsilon}}{R^2} + y^2 \cdot \frac{\abs{\varepsilon}}{R^2} +
       z'^2 \cdot \textrm{sign}(\varepsilon) \cdot \frac{\varepsilon^2}{R^2} - \textrm{sign}(\varepsilon)
       = 0
\end{split} \end{equation}

Let,
\begin{equation} \begin{cases}
a^2 = \frac{R^2}{\abs{\varepsilon}} \\
b^2 = \frac{R^2}{\varepsilon^2} \\
\epsilon = \textrm{sign}(\varepsilon)
\end{cases} \end{equation}

A readable \emph{normal form} is:

\begin{equation}
\frac{x^2}{a^2} + \frac{y^2}{a^2} + \epsilon \cdot \frac{z'^2}{b^2} - \epsilon
  = 0
\end{equation}

The case $k < -1$ gives the form,

\begin{equation}
\frac{x^2}{a^2} + \frac{y^2}{a^2} - \frac{z'^2}{b^2} = -1
\end{equation}

which is a \emph{hyperboloid of revolution of two sheets}.

The case $k > -1$ gives the form,

\begin{equation}
\frac{x^2}{a^2} + \frac{y^2}{a^2} + \frac{z'^2}{b^2} = 1
\end{equation}

which is a \emph{spheroid}. We note that for $k=0$, we have
$\varepsilon=1$ and thus $a=b$, which describes a \emph{sphere}.

\subsubsection{Rationale for defining the shape from quadrics}
We explore the rationale which leads from the general quadric (with potentially
two sheets) to the \lstinline{standard} altitude definition.

The general quadric is defined as,

\begin{equation}
x^2 + y^2 + (k+1) \cdot z^2 - 2 R \cdot z = 0
\end{equation}

We solve for $z$.

In the $k=-1$ case, the quadric is a circular paraboloid and has a single
sheet, thus the solution is unambiguous.

\begin{equation}
z = \frac{c \cdot r^2}{2}
\end{equation}

In the $k\neq-1$ cases, the solution is given by solving the quadratic equation.

\begin{equation} \begin{cases}
\Delta = 4 R^2 - 4 (k+1) \cdot r^2 \\
z = \frac{2R \pm \sqrt{\Delta}}{2(k+1)}
\end{cases} \end{equation}

We have two solutions given by,

\begin{equation}
z = \frac{R \pm \sqrt{R^2 - (k+1) \cdot r^2}}{k + 1}
\end{equation}

These define the two sheets of the spheroid or the hyperboloid of revolution of
two sheets. An explicit altitude formula may only contain a single sheet. We
choose to have $z(r=0) = 0$.

\begin{equation} \begin{split}
& z(r=0) = 0 \\
\iff & \frac{R \pm \sqrt{R^2}}{k+1} = 0 \\
\iff & R \pm \abs{R} = 0
\end{split} \end{equation}

We have to distinguish cases based on the sign of $R$.
\begin{itemize}
\item $R \geq 0$: we choose the solution with minus sign in order to constrain
                  $z(r=0)=0$,
\item $R < 0$: we choose the solution with plus sign.
\end{itemize}

Therefore, we build our solution as:
\begin{equation} \begin{split}
& z = \frac{R - \textrm{sign}(R) \cdot \sqrt{R^2 - (k+1) \cdot r^2}}{k+1} \\
\iff & (k+1) \cdot z = R - \textrm{sign}(R) \cdot \sqrt{R^2 - (k+1) \cdot r^2}\\
\iff & (k+1) \cdot z \cdot \left( R + \textrm{sign}(R) \cdot
       \sqrt{R^2 - (k+1) \cdot r^2} \right) = R^2 - (R^2 - (k+1) \cdot r^2) \\
\iff & z \cdot \left( R + \textrm{sign}(R) \cdot
       \sqrt{R^2 - (k+1) \cdot r^2} \right) = r^2 \\
\iff & z = \frac{r^2}{R + \textrm{sign}(R) \cdot \sqrt{R^2 - (k+1) \cdot r^2}}\\
\iff & z = \frac{r^2}{R + \textrm{sign}(R) \cdot \abs{R} \cdot
                      \sqrt{1 - (k+1) \cdot r^2 \cdot c^2}} \\
\iff & z = \frac{c \cdot r^2}{1 + \sqrt{1 - (k+1) \cdot r^2 \cdot c^2}}
\end{split} \end{equation}

Which is the \lstinline{standard} altitude formula in use. We see the $k=-1$
case is still compatible with this formula.

\subsubsection{Intersection formula}
The intersection of a ray with a \lstinline{standard} shape may be expressed
in closed-form. Finding it involves substituting the usual ray equation in
the \lstinline{standard} surface expression.

The ray equation being, as usual,

\begin{equation}
\begin{bmatrix} x \\ y \\ z \end{bmatrix} =
\begin{bmatrix} x_P \\ y_P \\ 0 \end{bmatrix} + t \cdot
\begin{bmatrix} l \\ m \\ n \end{bmatrix}
\end{equation}

and the quadric implicit equation,

\begin{equation}
x^2 + y^2 + (k+1) \cdot z^2 - 2R \cdot z = 0
\end{equation}

Finding the expression for solutions is easy, but finding an expression which
is numerically well behaved is more involved. Thankfully, Welford found such
an expression \cite{Welford:1986}, which we use with a slight modification.

\begin{equation} \begin{cases}
f = c \cdot ({x_P}^2 + {y_P}^2) \\
g = n - c \cdot (l \cdot x_P + m \cdot y_P) \\
t = \frac{f}{g + \textrm{sign}(n) \cdot
             \sqrt{g^2 - c \cdot f \cdot (1 + k \cdot n^2)}}
\end{cases} \end{equation}

Our minor modification is the addition of $\textrm{sign}(n)$ which allows
choosing the right root regardless of ray orientation.

The case where $g^2 - c \cdot f \cdot (1 + k \cdot n^2) \leq 0$ corresponds
to an error case of absence of intersection or ray tangency.

In non-error cases, the intersection point is given as usual with:

\begin{equation}
I = P + t \cdot V
\end{equation}

The hand validation being laborious, we validated the intersection formula with
a symbolic math tool.

\subsubsection{Normal vector}
The normal vector at some point on the surface is found by computing the
gradient of the implicit quadric expression, this is the method followed by
Welford \cite{Welford:1986}.

Let $\varepsilon = k+1$, the implicit quadric expression is:

\begin{equation}
F(x,y,z) = \frac{c}{2} \cdot (x^2 + y^2 + \varepsilon \cdot z^2) - z = 0
\end{equation}

We compute the gradient:

\begin{equation} \begin{cases}
\frac{\partial F}{\partial x} = c \cdot x\\
\frac{\partial F}{\partial y} = c \cdot y\\
\frac{\partial F}{\partial z} = c \cdot \varepsilon \cdot z - 1\\
\end{cases} \end{equation}

The norm of the unit vector is then,

\begin{equation}
\abs{\overrightarrow{N}} =
\sqrt{c^2 \cdot (x^2 + y^2) + c^2 \cdot \varepsilon^2
  \cdot z^2 + 1 - 2 c \cdot \varepsilon \cdot z} \\
\end{equation}

We know from the quadric expression that,

\begin{equation}
x^2 + y^2 = \frac{2}{c} \cdot z - \varepsilon \cdot z^2
\end{equation}

Hence,

\begin{equation} \begin{split}
\abs{\overrightarrow{N}} &=
\sqrt{c^2 \cdot \left( \frac{2}{c} \cdot z - \varepsilon \cdot z^2 \right)
  + c^2 \cdot \varepsilon^2 \cdot z^2 + 1 - 2 c \cdot \varepsilon \cdot z} \\
&= \sqrt{1 - 2 c \cdot (\varepsilon - 1) \cdot z + c^2 \cdot \varepsilon \cdot
         (\varepsilon - 1) \cdot z^2}
\end{split} \end{equation}

The unit vector with correct orientation is then:

\begin{equation}
\overrightarrow{N} = -
\begin{bmatrix}
c \cdot x \\ c \cdot y \\ c \cdot \varepsilon \cdot z - 1
\end{bmatrix} \cdot
\frac{\textrm{sign}(n) \cdot \textrm{sign}(c \cdot \varepsilon \cdot z - 1)}
{\abs{\overrightarrow{N}}}
\end{equation}

Let $D = \varepsilon \cdot c \cdot z - 1$, we can simplify the expression
of $\overrightarrow{N}$ if we know this sign. Let's determine it.

Looking at the implicit altitude formula \cref{eq:standard-z}, we see
$z$ and $c$ are of the same sign, so we can write,

\begin{equation}
D = \varepsilon \cdot \abs{c} \cdot \abs{z} - 1
\end{equation}

We can express the range of validity of $\varepsilon$ in the definition
of the \lstinline{standard} shape.

\begin{equation} \begin{split}
& 1 - (k+1) \cdot c^2 \cdot r^2 \geq 0 \\
\iff & 1 - \varepsilon \cdot c^2 \cdot r^2 \geq 0 \\
\iff & \varepsilon \leq \frac{1}{c^2 \cdot r^2}
\end{split} \end{equation}

Plugging this limitation into the expression of $D$, we get,

\begin{equation} \begin{split}
D &\leq \frac{1}{c^2 \cdot r^2} \cdot \abs{c} \cdot \abs{z} - 1 \\
  &\leq \frac{1}{\abs{c}} \cdot \frac{\abs{z}}{r^2} - 1
\end{split} \end{equation}

By the surface definition \cref{eq:standard-z}, we have,

\begin{equation}
\abs{z} \leq \abs{c} \cdot r^2
\end{equation}

Thus,

\begin{equation} \begin{split}
D &\leq \frac{\abs{c} \cdot r^2}{\abs{c} \cdot r^2} -1 \\
  &\leq 0
\end{split} \end{equation}

The simplified expression for the unit normal vector opposite to the incoming
ray is then,

\begin{equation}
\overrightarrow{N} =
\begin{bmatrix}
c \cdot x \\ c \cdot y \\ c \cdot \varepsilon \cdot z - 1
\end{bmatrix} \cdot
\frac{\textrm{sign}(n)}{
\sqrt{1 - 2 c \cdot (\varepsilon - 1) \cdot z + c^2 \cdot \varepsilon \cdot
         (\varepsilon - 1) \cdot z^2}}
\end{equation}

