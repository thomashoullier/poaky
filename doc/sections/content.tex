\section{Scope}
Poaky deals with tracing rays of light through optical systems,
under the laws of geometrical optics. It allows defining an optical system
with surfaces propagates rays through the system. Poaky reports the
state of rays at given optical surfaces.

The following optical design related items are left out:
\begin{itemize}
\item Phase computation: The rays are purely described by their trajectory.
\item Building objective functions and constraints: This belongs in a higher
layer which uses the states of rays to build performance criteria.
\item Optimization: This is an even higher layer than objective functions.
\item Ray-aiming: We can implement this on top of the current program.
\item User-level optical system definition: The optical system defined
      by the user does not use the same formalism as that of the current
      program.
\end{itemize}

\textcolor{red}{Whether ray-aiming and higher functions will be included
in the current repository or another is not yet decided.}

\section{Architecture}
The software architecture follows an object-oriented, bottom-up approach.
From the lowest level of abstraction to highest, we may describe the
major objects as follows:

\begin{itemize}
\item \lstinline{ray}: Rays are the most fundamental data object in the
program.  Each ray is represented by $(x, y , z)$ local coordinates and $(l, m,
n)$ local direction cosines.
\item \lstinline{ray operations}: These are low-level operations modifying
\lstinline{rays}.  They are used by \lstinline{low-level surfaces}. They may be
common to multiple surface types. A typical example is the intersection of a
ray with a sphere.
\item \lstinline{low-level surfaces}: These are low-level objects which are
ordered in succession. These are groupings of \lstinline{ray operations} which
correspond to common optical design surfaces. There are two fundamental types:
\begin{itemize}
\item \lstinline{surface}: This group of operations takes a ray starting from
the local surface plane and outputs rays also in the local coordinate system at
the computed positions.
\item \lstinline{transfer}: These operations take rays from a starting surface
coordinate system and propagate them in straight lines to another plane.  The
output rays are expressed in a new local surface plane.
\end{itemize}
\item \lstinline{ray bundles}: These are arrays of \lstinline{rays}. They
include some higher level utilities such as tracking and reporting the state of
rays.
\item \lstinline{low-level system}: This is a succession of
\lstinline{low-level surfaces}.
\item \lstinline{low-level tracer}: This object holds both a
\lstinline{low-level system} and a \lstinline{ray bundle}. It is responsible
for applying operations and reporting the state of the rays to higher parts
of the program.
\end{itemize}

\textcolor{red}{TODO: Summarize with a graphviz diagram.}
