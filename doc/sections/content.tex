\section{Scope}
Poaky deals with tracing rays of light through optical systems,
under the laws of geometrical optics. It allows defining an optical system
with surfaces propagates rays through the system. Poaky reports the
state of rays at given optical surfaces.

The following optical design related items are left out:
\begin{itemize}
\item Phase computation: The rays are purely described by their trajectory.
\item Building objective functions and constraints: This belongs in a higher
layer which uses the states of rays to build performance criteria.
\item Optimization: This is an even higher layer than objective functions.
\item Ray-aiming: We can implement this on top of the current program.
\item User-level optical system definition: The optical system defined
      by the user does not use the same formalism as that of the current
      program.
\end{itemize}

Many concepts about raytracing have been explored in depth in a previous
work \cite{Houllier-thesis}.

\textcolor{red}{Whether ray-aiming and higher functions will be included
in the current repository or another is not yet decided.}

\section{Architecture}
The software architecture follows an object-oriented, bottom-up approach.
From the lowest level of abstraction to highest, we may describe the
major objects as follows:

\begin{itemize}
\item \lstinline{ray}: Rays are the most fundamental data object in the
program.  Each ray is represented by $(x, y, z)$ local coordinates and $(l, m,
n)$ local direction cosines.
\item \lstinline{rop} (ray operations): These are low-level operations
modifying \lstinline{rays}.  They are used by \lstinline{low-level surfaces}.
They may be common to multiple surface types. A typical example is the
Snell law.
\item \lstinline{shape}: These are geometric surface shapes. They are used
mainly for specifying intersection methods with rays, and respond to queries
about normal vector and altitude at given $(x, y)$ coordinates.
\item \lstinline{lpart} (low-level part): These are low-level objects which
are ordered in succession. These are groupings of \lstinline{rop}
which correspond to common optical design surface elements. There are two
fundamental subtypes:
\begin{itemize}
\item \lstinline{surf} (surface): This group of operations takes a ray starting
from the local surface plane and outputs rays also in the local coordinate
system at the computed positions. May inherit a \lstinline{shape}.
\item \lstinline{tfr} (transfer): These operations take rays from a starting
surface coordinate system and propagate them in straight lines to another
plane. The output rays are expressed in a new local surface plane.
\end{itemize}
\item \lstinline{bun} (ray bundles): These are arrays of \lstinline{ray}. They
include some higher level utilities such as tracking and reporting the state of
rays.
\item \lstinline{lsys} (low-level system): This is a succession of
\lstinline{lpart}.
\item \lstinline{ltrac} (low-level tracer): This object holds both a
\lstinline{lsys} and a \lstinline{bun}. It is responsible for applying
operations and reporting the state of the rays to higher parts of the program.
\end{itemize}

This architecture is summarized informally using a diagram
(\cref{fig:arch-overview}).

\begin{figure} 
\includesvg[width=.9\textwidth]{images/arch/overview/overview.svg}
\caption{\label{fig:arch-overview} Architecture overview.}
\end{figure}

\section{Functional description}

\textcolor{red}{TODO: \begin{itemize}
\item Detail the behavior of a minimal set of components for
only 3D mirrors.
\end{itemize}}

\subsection{base}
Some base types are useful throughout the program. These are detailed in this
section.

\subsubsection{Point3}
\lstinline{Point3} are points in 3D space. They are described by $(x, y, z)$
coordinates.

\subsubsection{UVec3}
\lstinline{UVec3} are unit vectors in the direction of the propagation
of light. These may also be refined to as \emph{direction cosines},
as their components are $(l, m, n)$.

\textcolor{red}{Give some formulae.}

\subsection{ray}

\textcolor{red}{xyzlmn description, short rationale on the fact rays hold
minimal information and their meaning depends on context.
Error codes for each ray?}

\subsection{rop}

\textcolor{red}{
\begin{itemize}
\item transfer operation
\item reflection off mirrors
\end{itemize}
Explain rationale to have the most specialized operations possible here.
And why operations are dissociated from lpart.}

\subsection{shape}

\textcolor{red}{\begin{itemize}
\item ray/plane intersection
\item ray/sphere intersection
\item sphere normal vector
\end{itemize}}

\subsection{lpart}

\subsubsection{tfr}

\subsubsection{surf}
\textcolor{red}{
\begin{itemize}
\item Plane (which does nothing)
\item Plane mirror
\item Spherical mirror
\end{itemize}}

\section{Tests and benchmarks}
We document what tests are performed on the components of the software.
We also detail a representative performance report.

\subsection{Tests}

\subsection{Performance report}
A performance report helps both the user and developer understand the strengths
and weaknesses of the software computations.
