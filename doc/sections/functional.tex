\section{Functional description}
This section defines the program's objects and their associated
operations. The style is minimal and close to the computations. For
the rationale sustaining the computation and complementary information,
see the justification section (\cref{sec:justification}).

\textcolor{red}{TODO: \begin{itemize}
\item Detail the behavior of a minimal set of components for
      only 3D mirrors.
\item In a separate section, detail only the abstract methods that
      are defined?
\end{itemize}}

\subsection{base}
Some base types are useful throughout the program. These are detailed in this
section.

\subsubsection{Point3}
\lstinline{Point3} are points in 3D space. They are described by $(x, y, z)$
coordinates.

\subsubsection{UVec3}
\lstinline{UVec3} are unit vectors in 3D space.

\subsection{ray}
\lstinline{ray} objects are the centerpiece of the simulation. They must be
lightweight objects.  \lstinline{ray} holds a position and a unit vector in the
direction and orientation of the propagation of light:

\begin{itemize}
\item \lstinline{Point3 p}: A point.
\item \lstinline{UVec3 v}: A vector, oriented by light propagation.
\end{itemize}

The interpretation of the data contained in a \lstinline{ray} is dependent
on context, as they are expressed in the current surface coordinate system.

In addition to their geometric definition, rays also hold a status code.
This code signals whether raytracing operations were successful, and if
not, which error case was encountered.

\begin{itemize}
\item \lstinline{int code}: Status code.
\end{itemize}

The status codes are defined in \cref{tab:ray-status-codes}.

\begin{table} \caption{\label{tab:ray-status-codes} Ray status codes.}
\begin{tabular}{| c | l |} \hline
\textbf{Code} & \textbf{Meaning} \\ \hline
0 & Success, the ray is valid.\\ \hline
1 & Sphere intersection: No intersection. \\ \hline
2 & Sphere intersection: Intersection beyond first hemisphere. \\
\hline \end{tabular}
\end{table}

\subsection{rop}

\textcolor{red}{
\begin{itemize}
\item transfer operation
\item reflection off mirrors
\end{itemize}
Explain rationale to have the most specialized operations possible here.
And why operations are dissociated from lpart.}

\subsection{shape}
\lstinline{shape} objects specify an intersection operation with rays.
The intersection operation takes a \lstinline{ray} expressed in the current
surface coordinate system with point on the local plane. It propagates
the ray until it hits the first encountered part of the shape. It modifies
the ray in-place. The modified ray is still expressed in the same current
coordinate system.

Subtypes are defined.

\textcolor{red}{\begin{itemize}
\item Describe the virtual shape methods.
\end{itemize}}

\subsubsection{plane}
\paragraph{Definition}
A \lstinline{plane} is the local $z=0$ plane in the current \gls{LCS}.
It is specified implicitely.

\paragraph{intersect}
The input ray is already on the local plane. We do \emph{nothing} and cannot
fail.

\paragraph{normal}
\textcolor{red}{TODO}

\subsubsection{sphere}

\paragraph{Definition}
The \lstinline{sphere} is specified by:

\begin{itemize}
\item $R$: radius of curvature
\end{itemize}

The sphere is defined in the \gls{LCS} by \cref{eq:sphere-def}.

\begin{equation} \label{eq:sphere-def}
x^2 + y^2 + (z - R)^2 = R^2
\end{equation}

$R$ is the signed distance $\overline{AC}$, with $C$ the sphere center
(\cref{fig:sphere-def-lcs}). Only the hemisphere closest to the local plane
is considered as part of the \lstinline{sphere} shape.

\begin{figure} \caption{\label{fig:sphere-def-lcs} Sphere definition
in the LCS. Both signs of $R$ are represented.}
\includesvg[height=.2\textheight, width=.9\textwidth, keepaspectratio]
           {images/shape/sphere.svg}
\end{figure}

\textcolor{red}{TODO:
1. What happens if we define a sphere with R = 0? and for small R?
}

\paragraph{intersect}
The intersection between the input ray and the sphere is computed as
follows. The quantities involved are illustrated on \cref{fig:sphere-inter}.

\begin{figure} \caption{\label{fig:sphere-inter} Sphere intersection with
a ray diagram. The intersection point is $I$. The diagram is drawn arbitrarily
in the case $R<0$.}
\includesvg[height=.2\textheight, width=.9\textwidth, keepaspectratio]
           {images/shape/sphere-rayinter.svg}
\end{figure}

\begin{equation} \label{eq:ray-sphere-inter1}
\begin{cases}
b &= 2 (x \cdot l + y \cdot m - n \cdot R) \\
c &= x^2 + y^2 \\
\Delta &= b^2 - 4 c
\end{cases}
\end{equation}

If $\Delta \leq 0$, then no intersection exists. This is a ray error case.
Else, we continue with \cref{eq:ray-sphere-inter-tsol}.

\begin{equation} \label{eq:ray-sphere-inter-tsol}
t_\textrm{sol} = \frac{-b + \textrm{sign}(b) \cdot \sqrt{\Delta}}{2}
\end{equation}

With:

\begin{equation}
\textrm{sign}(x) = \begin{cases}
1 & \text{if } x \geq 0 \\
-1 & \text{otherwise}
\end{cases}
\end{equation}

Let $(x_I, y_I, z_I)$ the intersection point we wish to compute.

\begin{equation}
z_I = t_\textrm{sol} \cdot n
\end{equation}

If $\abs{z_I} \geq \abs{R}$, we have an intersection in the hemisphere furthest
from the local plane. This is another ray error case. Else, the remainder of
the solution can be computed.

\begin{equation}
\begin{aligned}
x_I &= x + t_\textrm{sol} \cdot l \\
y_I &= y + t_\textrm{sol} \cdot m
\end{aligned}
\end{equation}

\paragraph{normal}
The unit vector $\overrightarrow{N}$ normal to the surface of the sphere at
the query position $(x_q, y_q)$, and with positive $z$ LCS coordinate, is
computed as follows.

\textcolor{red}{TODO:
* Do we always want N pointed towards +Z?}

\subsection{lpart}

\subsubsection{tfr}

\subsubsection{surf}
\textcolor{red}{
\begin{itemize}
\item Plane (which does nothing)
\item Plane mirror
\item Spherical mirror
\end{itemize}}

