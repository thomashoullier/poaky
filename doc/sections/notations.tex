\section{Notations}
\subsection{Multiplication and dot product}
The symbol $\cdot$ may refer to:
\begin{itemize}
\item Scalar multiplication
\item Scalar/Vector or Scalar/Matrix multiplication
\item Vector dot product
\end{itemize}

\subsection{Absolute value and 2-norm}
The unary operator notation $\abs{\cdot}$ may refer to:
\begin{itemize}
\item Absolute value of a scalar
\item 2-norm (or \emph{Euclidian} norm) of a vector
\end{itemize}

\subsection{Sign function}
We use a sign function, defined as in \cref{eq:sign-fun}.

\begin{equation} \label{eq:sign-fun}
\textrm{sign}(x) = \begin{cases}
1 & \text{if } x \geq 0 \\
-1 & \text{otherwise}
\end{cases} \end{equation}

\section{Glossary}
Some technical terms are recurring. They are specific either to this document
or to optical design and thus must be defined.

\subsection{Local plane}
The \emph{local plane} is the $z=0$ plane in a \gls{LCS} (\cref{sec:LCS}).

\subsection{Concave or Convex}
We must disambiguate what we mean by \emph{convex} and \emph{concave}, as other
conventions are used in other fields. These adjectives apply to surface
shapes, often in the context of raytracing as seen by incoming rays.
It refers to the sign of the curvature of the shape.  The two terms are
illustrated on \cref{fig:concave-convex}. A surface shape may be described as
convex/concave either globally or locally at the ray intersection point.

\begin{figure} \caption{\label{fig:concave-convex} Illustration of a
ray incoming onto a concave surface (left), and on a convex surface (right).}
\includesvg[height=.2\textheight, width=.9\textwidth, keepaspectratio]
           {images/glossary/concave-convex.svg}
\end{figure}
